\documentclass[unknownkeysallowed]{beamer}
\mode<presentation>
{
%  \usetheme{AnnArbor}
%  \usetheme{Dresden}
%  \usetheme{Montpellier}
%  \usetheme{Antibes}
%  \usetheme{Frankfurt}
%  \usetheme{PaloAlto}
%  \usetheme{Bergen}
%  \usetheme{Boadilla}
%  \usetheme{Goettingen}
%  \usetheme{Pittsburgh}	%!!
%  \usetheme{Berkeley}
%  \usetheme{Hannover}
%  \usetheme{Rochester}		%!!!
%  \usetheme{Berlin}
%  \usetheme{Ilmenau}
%  \usetheme{Singapore}
  \usetheme{Boadilla}		%viel platz
%  \usetheme{JuanLesPins}
%  \usetheme{Szeged}		%!
%  \usetheme{boxes}
%  \usetheme{Luebeck}
%  \usetheme{Warsaw}
%  \usetheme{Copenhagen}
%  \usetheme{Madrid}
%  \usetheme{Darmstadt}
%  \usetheme{Malmoe}
%  \usetheme{default}
%  \usetheme{JuanLesPins}

%  \usetheme{Marburg}


%\usefonttheme{professionalfonts}
%	default | professionalfonts | serif |
%	structurebold | structureitalicserif |
%	structuresmallcapsserif
%\useinnertheme{rounded}
%	circles | default | inmargin |
%	rectangles | rounded

%  \setbeamercovered{transparent}
  % oder auch nicht


\usecolortheme{rose}


\definecolor{uaf yellow}{cmyk}{0,0.16,1,0} % official UAF yellow
\definecolor{light yellow}{cmyk}{0.01,0,0.16,0}
\definecolor{uaf blue}{cmyk}{1,0.66,0,0.02} % official UAF blue
\definecolor{light blue}{cmyk}{0.22,0.11,0,0}
\definecolor{arsc blue}{HTML}{005496}
\definecolor{arsc red}{HTML}{a20a42}
\definecolor{arsc green}{HTML}{009a82}
\definecolor{light gray}{HTML}{777777}

  %navigation aus, klaut nur platz
  \setbeamertemplate{navigation symbols}{}
% Reset title background to default
%\setbeamertemplate{title page}[default]
\setbeamercolor{title}{bg=}
\setbeamercolor{frametitle}{bg=uaf blue, fg=white}
\setbeamercolor{institute}{fg=white}
\setbeamercolor{date}{fg=white}
\setbeamercolor{block}{bg=}
%\setbeamercolor{title}{fg=black}

% Reset block background to default
%\setbeamertemplate{blocks}[default]
%\setbeamercolor{block title}{bg=}
%\setbeamercolor{block body}{bg=}

\beamertemplatenavigationsymbolsempty  
\setbeamertemplate{blocks}[rounded][shadows=false]

\useinnertheme{circles}

}
\usepackage[latin1]{inputenc}
\usepackage{latexsym}
\usepackage{amsfonts}
%\usepackage{natbib}
\usepackage{fancyhdr}
\usepackage{graphicx}
%\usepackage{subfigure}
% oder was auch immer
\usepackage{grffile}
\usepackage{pgf}
\usepackage{tikz}

\usepackage[latin1]{inputenc}
% oder was auch immer
%\usepackage{media9}
\usepackage{times}
\usepackage[T1]{fontenc}
%\usepackage{appendixnumber}
% Oder was auch immer. Zu beachten ist, das Font und Encoding passen
% m�ssen. Falls T1 nicht funktioniert, kann man versuchen, die Zeile
% mit fontenc zu l�schen.

\hypersetup{
    bookmarks=true,         % show bookmarks bar?
    unicode=false,          % non-Latin characters in Acrobat's bookmarks
    pdftoolbar=true,        % show Acrobat's toolbar?
    pdfmenubar=true,        % show Acrobat's menu?
    pdffitwindow=false,     % window fit to page when opened
    pdfstartview={FitH},    % fits the width of the page to the window
    pdftitle={My title},    % title
    pdfauthor={Author},     % author
    pdfsubject={Subject},   % subject of the document
    pdfcreator={Creator},   % creator of the document
    pdfproducer={Producer}, % producer of the document
    pdfkeywords={keyword1} {key2} {key3}, % list of keywords
    pdfnewwindow=true,      % links in new window
    colorlinks=false,       % false: boxed links; true: colored links
    linkcolor=red,          % color of internal links
    citecolor=green,        % color of links to bibliography
    filecolor=magenta,      % color of file links
    urlcolor=cyan           % color of external links
}

\title[Geodetic Methods]% (optional, nur bei langen Titeln n�tig)
{GEOS F431 / F631\\
Foundations of Geophysics\\[20pt]
-- Strain --
}

\author[Grapenthin]% (optional, nur bei vielen Autoren)
{Ronni Grapenthin\\
rgrapenthin@alaska.edu\\
Elvey 413B\\
(907) 474-7286}
% - Namen m�ssen in derselben Reihenfolge wie im Papier erscheinen.
% - Der \inst{?} Befehl sollte nur verwendet werden, wenn die Autoren
%   unterschiedlichen Instituten angeh�ren.

\institute[UAF-GI] % (optional, aber oft n�tig)
{}
% - Der \inst{?} Befehl sollte nur verwendet werden, wenn die Autoren
%   unterschiedlichen Instituten angeh�ren.
% - Keep it simple, niemand interessiert sich f�r die genau Adresse.

% - Namen m�ssen in derselben Reihenfolge wie im Papier erscheinen.
% - Der \inst{?} Befehl sollte nur verwendet werden, wenn die Autoren
%   unterschiedlichen Instituten angeh�ren.

% - Der \inst{?} Befehl sollte nur verwendet werden, wenn die Autoren
%   unterschiedlichen Instituten angeh�ren.
% - Keep it simple, niemand interessiert sich f�r die genau Adresse.

\date[September 08, 2020] % (optional, sollte der abgek�rzte Konferenzname sein)
{September 08, 2020}

% - Volle oder abgek�rzter Name sind m�glich.
% - Dieser Eintrag ist nicht f�r das Publikum gedacht (das wei�
%   n�mlich, bei welcher Konferenz es ist), sondern f�r Leute, die diese
%   Folien sp�ter lesen.

%\AtBeginSection[]
%{
%  \begin{frame}<beamer>
%    \frametitle{Outline}
%    \tableofcontents[currentsection,currentsubsection]
%  \end{frame}
%}

% Falls Aufz�hlungen immer schrittweise gezeigt werden sollen, kann
% folgendes Kommando benutzt werden:

%\beamerdefaultoverlayspecification{<+->}

%%switch on to have only frame numbers
\setbeamertemplate{footline}[frame number]

\defbeamertemplate*{title page}{customized}[1][]
{
		\begin{tikzpicture}
			\node[text width=\textwidth,
				fill=gray!70, 
				fill opacity=0.75,
				text opacity=1,
				rounded corners = 10pt,
				inner sep=2pt]{
				\begin{center}	
			  \usebeamerfont{title}{\bf \usebeamercolor[fg]{title} \inserttitle}
			  \par
			  \usebeamerfont{subtitle}\insertsubtitle\par
			  \bigskip
			  \usebeamerfont{author}\insertauthor\par
			  \bigskip
			  \usebeamerfont{institute}\insertinstitute\par
			  \bigskip
			  \usebeamerfont{date}\insertdate\par
			  \end{center}
			  };
	\end{tikzpicture}		  
%	\vspace{0.4cm}\usebeamercolor[fg]{titlegraphic}\inserttitlegraphic 
%	\begin{flushright}
%	\vspace{-1.25cm}\includegraphics[width=2cm]{../moore_logo_transp.png}\vspace{5cm}
%	\end{flushright}
}

\begin{document}
\setbeamertemplate{background}{\includegraphics[height=\paperheight]{/home/roon/Pictures/cliff.jpg}}

	\begin{frame}
	\begin{center}
		\titlepage
	\end{center}
	\end{frame}


\setbeamertemplate{background}{}

\setbeamertemplate{background}{}

\begin{frame}
	\frametitle{Deformation}
	\begin{center}
		\includegraphics[width=.75\textwidth]{/home/roon/work/teaching/2015/Geodetic_methods/figures/lecture20_deformation.png}\\
		deformation = {\color[rgb]{1,0,0} translation} + {\color[rgb]{0,0,1} rotation} + {\color[rgb]{0.5, 0.5, 0.5} dilatation} 
	\end{center}
	\begin{itemize}
		\item translation, rotation: rigid body deformation (angles, volume preserved)
		\item dilatation: volume changes, angles change
	\end{itemize}
\end{frame}


\begin{frame}
	\frametitle{Transformations: Translation}
	\begin{center}
		\includegraphics[width=.6\textwidth]{/home/roon/work/teaching/2015/Geodetic_methods/figures/lecture20_translation.png}
	\end{center}
	\begin{flushright}
	\tiny{\emph{J. Freymueller}}
	\end{flushright}
	\uncover<2->{
		\begin{eqnarray}
		x'_1 & = & x_1 + \Delta x_1 \nonumber\\
		x'_2 & = & x_2 + \Delta x_2 \nonumber
		\end{eqnarray}
		(indices indicate vector components!)
	}
\end{frame}

\begin{frame}
	\frametitle{Transformations: Rotation}
	\begin{center}
		\includegraphics<1-2>[width=.6\textwidth]{/home/roon/work/teaching/2015/Geodetic_methods/figures/lecture20_rotation1.png}
		\includegraphics<3->[width=.6\textwidth]{/home/roon/work/teaching/2015/Geodetic_methods/figures/lecture20_rotation2.png}
	\end{center}
	\begin{flushright}
	\tiny{\emph{J. Freymueller}}
	\end{flushright}
	\uncover<2->{
		\vspace{-0.5cm}
		Given 2 systems, how are vector components related?
	}
	\uncover<4->{
		\begin{columns}
			\column{0.5\textwidth}
			\begin{eqnarray}
				x_1 & = & x'_1 cos(\theta) - x'_2 sin(\theta) \nonumber \\
				x_2 & = & x'_1 sin(\theta) + x'_2 cos(\theta) \nonumber
			\end{eqnarray}
			\column{0.5\textwidth}
			\begin{eqnarray}
				x'_1 & = & x_1 cos(\theta)  + x_2 sin(\theta) \nonumber \\
				x'_2 & = & -x_1 sin(\theta) + x_2 cos(\theta) \nonumber
			\end{eqnarray}
		\end{columns}
	}
\end{frame}

\begin{frame}
	\frametitle{Transformations: Rotation}
	\begin{center}
		\includegraphics[width=.6\textwidth]{/home/roon/work/teaching/2015/Geodetic_methods/figures/lecture20_rotation2.png}
	\end{center}
	\begin{flushright}
	\tiny{\emph{J. Freymueller}}
	\end{flushright}
		\vspace{-0.5cm}
		Given 2 systems, how are vector components related?
		\begin{columns}
			\column{0.5\textwidth}
			\begin{eqnarray}
				{\bf x} & = & \left[ \begin{array}{cc}
						 cos(\theta) & -sin(\theta) \\
						 sin(\theta) & cos(\theta)
						 \end{array} \right] 
						 \left[ \begin{array}{c}
						 x'_1 \\
						 x'_2 \end{array} \right] \nonumber
			\end{eqnarray}
			\column{0.5\textwidth}
			\begin{eqnarray}
				{\bf x'} & = & \left[ \begin{array}{cc}
						 cos(\theta) & sin(\theta) \\
						 - sin(\theta) & cos(\theta)
						 \end{array} \right] 
						 \left[ \begin{array}{c}
						 x_1 \\
						 x_2 \end{array} \right] \nonumber
			\end{eqnarray}
		\end{columns}
\end{frame}

\begin{frame}
	\frametitle{Transformations: Rotation}
	\begin{center}
		\includegraphics[width=.6\textwidth]{/home/roon/work/teaching/2015/Geodetic_methods/figures/lecture20_rotation2.png}
	\end{center}
	\begin{flushright}
	\tiny{\emph{J. Freymueller}}
	\end{flushright}
	Rotating a vector is the same as rotating the coordinate system in the opposite direction
\end{frame}

\begin{frame}
	\frametitle{Transformations: Dilatation}
	\begin{center}
		\includegraphics[width=.9\textwidth]{/home/roon/work/teaching/2015/Geodetic_methods/figures/lecture20_dilatation.png}
	\end{center}
	\begin{flushright}
	\tiny{\emph{J. Freymueller}}
	\end{flushright}
	\vspace{-0.5cm}
	\uncover<2->{
			fractional length changes are {\bf normal strains}:
			\begin{eqnarray}
				\frac{change\_in\_length}{original\_length} & = & strain \nonumber \\
				du_1/X & = & \varepsilon_{xx}\nonumber \\
				du_2/Y & = & \varepsilon_{yy}\nonumber \\
				du_3/Z & = & \varepsilon_{zz}\nonumber
	 		\end{eqnarray}
	convention important: geologists often use positive = contraction, can be extension, too. Check!
	}
\end{frame}

\begin{frame}
	\frametitle{Transformations: Dilatation}
	\begin{center}
		\includegraphics[width=.9\textwidth]{/home/roon/work/teaching/2015/Geodetic_methods/figures/lecture20_dilatation.png}
	\end{center}
	\begin{flushright}
	\tiny{\emph{J. Freymueller}}
	\end{flushright}
		\vspace{-0.5cm}
		think in finite differences (infinitesimal lengths):
		\begin{eqnarray}
			\lim_{length \to 0}\frac{length - new\_length}{length} & = & derivative \nonumber \\
			\partial u_1/\partial x & = & \varepsilon_{xx}\nonumber \\
			\partial u_2/\partial y & = & \varepsilon_{yy}\nonumber \\
			\partial u_3/\partial z & = & \varepsilon_{zz}\nonumber
 		\end{eqnarray}
	convention important: geologists often use positive = contraction, can be extension, too. Check!
\end{frame}

\begin{frame}
	\frametitle{Transformations: Dilatation}
	\begin{center}
		\includegraphics[width=.5\textwidth]{/home/roon/work/teaching/2015/Geodetic_methods/figures/lecture20_dilatation.png}
	\end{center}
	\begin{flushright}
	\tiny{\emph{J. Freymueller}}
	\end{flushright}
	\vspace{-0.5cm}
	Dilatation ($\Delta$) defined as fractional volume change:
	\begin{eqnarray}
		\Delta & = & \frac{X(1+\varepsilon_{xx}) * Y(1+\varepsilon_{yy}) * Z(1+\varepsilon_{zz}) - X*Y*Z }{X*Y*Z} \nonumber\\
		       & = & \frac{X*Y*Z ((1+\varepsilon_{xx}) * (1+\varepsilon_{yy}) * (1+\varepsilon_{zz}) - 1) }{X*Y*Z} \nonumber\\
		       & = & (1+\varepsilon_{xx}) * (1+\varepsilon_{yy}) * (1+\varepsilon_{zz}) - 1 \nonumber
	\end{eqnarray}
	\uncover<2->{
	We use infinitesimal strain, products of strain can be dropped:
	\begin{eqnarray}
		\Delta & = & 1 + \varepsilon_{xx} + \varepsilon_{yy} + \varepsilon_{zz}  - 1 \nonumber \\
		       & = & \varepsilon_{xx} + \varepsilon_{yy} + \varepsilon_{zz}  \nonumber 
	\end{eqnarray}
	}
	\uncover<3->{
	seismic P waves are travelling oscillations of $\Delta$
	}
\end{frame}

\begin{frame}
	\frametitle{Strain: Normal Strain}
	\begin{center}
		\includegraphics[width=.9\textwidth]{/home/roon/work/teaching/2015/Geodetic_methods/figures/lecture20_dilatation.png}
	\end{center}
	\begin{flushright}
	\tiny{\emph{J. Freymueller}}
	\end{flushright}
		fractional length changes are {\bf normal strains}:
		\begin{eqnarray}
			\partial u_1/\partial x & = & \varepsilon_{xx}\nonumber \\
			\partial u_2/\partial y & = & \varepsilon_{yy}\nonumber \\
			\partial u_3/\partial z & = & \varepsilon_{zz}\nonumber
 		\end{eqnarray}
		components of strain proportional to derivatives of displacements in respective directions
\end{frame}

\begin{frame}
	\frametitle{Strain: Shear Strain}
	\begin{center}
		\includegraphics<1>[width=.65\textwidth]{/home/roon/work/teaching/2015/Geodetic_methods/figures/lecture20_shear_strain3.png}
		\includegraphics<2->[width=.65\textwidth]{/home/roon/work/teaching/2015/Geodetic_methods/figures/lecture20_shear_strain4.png}
	\end{center}
	\begin{flushright}
	\tiny{\emph{J. Freymueller}}
	\end{flushright}
	{\bf shear components of strain} measure change in shape / angles
	\begin{eqnarray}
		\varepsilon_{xy} & = & \varepsilon_{yx} = - \frac{1}{2} (\phi_1 + \phi_2) \nonumber
	\end{eqnarray}
	\uncover<2->{
	angles related to displacements (small angle approx: $tan(\phi) \approx \phi$):
	\begin{eqnarray}
		\phi_1 & = & tan(\phi_1) =  \frac{opposite}{adjacent} = - \frac{dy}{X} \nonumber \\
		\phi_2 & = & tan(\phi_2) =  \frac{opposite}{adjacent} = - \frac{dx}{Y} \nonumber
	\end{eqnarray}
	}
\end{frame}

\begin{frame}
	\frametitle{Strain: Shear Strain}
	\begin{center}
		\includegraphics<1>[width=.75\textwidth]{/home/roon/work/teaching/2015/Geodetic_methods/figures/lecture20_shear_strain4.png}
	\end{center}
	\begin{flushright}
	\tiny{\emph{J. Freymueller}}
	\end{flushright}
	{\bf shear components of strain} measure change in shape / angles
	\begin{eqnarray}
		\varepsilon_{xy} & = & \varepsilon_{yx} = \frac{1}{2} \left( \frac{\partial u_2}{\partial x} + \frac{\partial u_1}{\partial y} \right) \nonumber
	\end{eqnarray}
	subscripts: 1st -- direction normal to element, 2nd -- direction of shear
\end{frame}

\begin{frame}
	\frametitle{Strain: Shear Strain}
	\begin{center}
		\includegraphics<1>[width=.75\textwidth]{/home/roon/work/teaching/2015/Geodetic_methods/figures/lecture20_shear_strain2.png}
	\end{center}
	\begin{flushright}
	\tiny{\emph{J. Freymueller}}
	\end{flushright}
	{\bf shear strain} results in solid body rotation if $\phi_1 \neq \phi_2$:
	\begin{eqnarray}
		\varepsilon_{xy} & = & \varepsilon_{yx} = \frac{1}{2} \left( \frac{\partial u_2}{\partial x} + \frac{\partial u_1}{\partial y} \right) \nonumber
	\end{eqnarray}
\end{frame}

\begin{frame}
	\frametitle{Strain: Shear Strain}
	\begin{center}
		\includegraphics<1>[width=.75\textwidth]{/home/roon/work/teaching/2015/Geodetic_methods/figures/lecture20_shear_strain2.png}
	\end{center}
	\begin{flushright}
	\tiny{\emph{J. Freymueller}}
	\end{flushright}
	{\bf shear strain} results in solid body rotation if $\phi_1 \neq \phi_2:$
	\begin{eqnarray}
		\omega_{z} & = & - \frac{1}{2} \left( \phi_1 - \phi_2 \right) = \frac{1}{2} \left( \frac{\partial u_2}{\partial x} - \frac{\partial u_1}{\partial y} \right)\nonumber
	\end{eqnarray}
\end{frame}

\begin{frame}
	\frametitle{Strain: Shear Strain}
	\begin{center}
		\includegraphics<1>[width=.75\textwidth]{/home/roon/work/teaching/2015/Geodetic_methods/figures/lecture20_shear_strain2.png}
	\end{center}
	\begin{flushright}
	\tiny{\emph{J. Freymueller}}
	\end{flushright}
	\begin{itemize}
		\item if $\phi_1 = \phi_2$: no solid body rotation -- {\bf pure shear}
		\item if $\phi_1 = 0$: solid body rotation + shear -- {\bf simple shear} (strike slip faulting)
	\end{itemize}
\end{frame}

\begin{frame}
	\frametitle{Putting it all together}
	\begin{center}
		\includegraphics[width=.75\textwidth]{/home/roon/work/teaching/2015/Geodetic_methods/figures/lecture20_deformation.png}\\
	\end{center}
	\begin{eqnarray}
		deformation & = & {\color[rgb]{1,0,0} translation} + 
						  {\color[rgb]{0.5, 0.5, 0.5} dilatation} + 
						  {\color[rgb]{0,0,1} rotation} \nonumber \\
		u           & \approx & {\color[rgb]{1,0,0} x + dx}    + 
						  {\color[rgb]{0.5, 0.5, 0.5} \frac{1}{2} \left( \frac{\partial u_i}{\partial x_j} + \frac{\partial u_j}{\partial x_i} \right)} +
						  {\color[rgb]{0,0,1} \frac{1}{2} \left( \frac{\partial u_i}{\partial x_j} - \frac{\partial u_j}{\partial x_i} \right)} \nonumber
	\end{eqnarray}
%	correct formal description follows \dots
\end{frame}

\begin{frame}
	\frametitle{Deformation}
		\begin{center}
			\includegraphics<1>[width=.4\textwidth]{/home/roon/work/teaching/2015/Geodetic_methods/figures/lecture20_displacement1.png}
			\includegraphics<2>[width=.4\textwidth]{/home/roon/work/teaching/2015/Geodetic_methods/figures/lecture20_displacement2.png}
			\includegraphics<3->[width=.4\textwidth]{/home/roon/work/teaching/2015/Geodetic_methods/figures/lecture20_displacement3.png}
		\end{center}
		\begin{flushright}
		\tiny{\emph{J. Freymueller}}
		\end{flushright}
		\uncover<4->{
		use Taylor Series expansion to relate the two vectors:
		{\footnotesize
		\begin{eqnarray}
			u_i({\bf x_0} + {\bf dx}) & = & u_i({\bf x_0}) + 
											\left( \frac{\partial u_i}{\partial x_1} \right) dx_1 +
											\left( \frac{\partial u_i}{\partial x_2} \right) dx_2 +
											\left( \frac{\partial u_i}{\partial x_3} \right) dx_3 \nonumber
		\end{eqnarray}
		}
		}
		\uncover<5->{
		\noindent 3 equations: i=1,2,3\\
		first term: translation, remainder: rotation + dilatation\\
		9 values $\partial u_i / \partial x_j$ for $i,j = 1\dots3$
		}
\end{frame}

\begin{frame}
	\frametitle{Deformation Tensor}
		\begin{columns}
			\column{0.4\textwidth}
			\begin{center}
				\includegraphics[width=\textwidth]{/home/roon/work/teaching/2015/Geodetic_methods/figures/lecture20_displacement3.png}
			\end{center}
			\begin{flushright}
				\tiny{\emph{J. Freymueller}}
			\end{flushright}
			\column{0.6\textwidth}
			\begin{eqnarray}
				u({\bf x_0} + {\bf dx}) & = & u({\bf x_0})+ \left[ \begin{array}{ccc}
											\frac{\partial u_1}{\partial x_1} & \frac{\partial u_1}{\partial x_2} & \frac{\partial u_1}{\partial x_3} \\
											\frac{\partial u_2}{\partial x_1} & \frac{\partial u_2}{\partial x_2} & \frac{\partial u_2}{\partial x_3} \\
											\frac{\partial u_3}{\partial x_1} & \frac{\partial u_3}{\partial x_2} & \frac{\partial u_3}{\partial x_3} 
											\end{array} \right]\nonumber
			\end{eqnarray}
			\uncover<2->{
			\begin{itemize}
				\item matrix describes dilatation and rotation
				\item is a 2-direction  (rank 2) tensor: contains normal strain, and strain perpendicular to face on which it acts
%				\item think of tensors as extension of vectors (magnitude and direction), which are an extension of scalars (magnitude)
			\end{itemize}
			}
		\end{columns}
\end{frame}

\begin{frame}
	\frametitle{Separate Rotation and Strain}
		We can separate gradient tensor into the sum of two tensors: strain tensor and rotation tensor from:
		{\tiny
		\begin{eqnarray}
		deformation & = & {\color[rgb]{1,0,0} translation} + 
					  {\color[rgb]{0.5, 0.5, 0.5} dilatation / strain} + 
					  {\color[rgb]{0,0,1} rotation} \nonumber \\		
		\uncover<2->{
				u_i({\bf x_0} + {\bf dx}) & = & {\color[rgb]{1,0,0} u_i({\bf x_0})} + 
							  			{\color[rgb]{0.5, 0.5, 0.5} \frac{1}{2} \left( \frac{\partial u_i}{\partial x_j} + \frac{\partial u_j}{\partial x_i} \right) dx_j} +
										{\color[rgb]{0,0,1} \frac{1}{2} \left( \frac{\partial u_i}{\partial x_j} - \frac{\partial u_j}{\partial x_i} \right) dx_j}\nonumber \\
		}
		\uncover<3->{
		\left[ \begin{array}{ccc}
			\frac{\partial u_1}{\partial x_1} & \frac{\partial u_1}{\partial x_2} & \frac{\partial u_1}{\partial x_3} \\
			\frac{\partial u_2}{\partial x_1} & \frac{\partial u_2}{\partial x_2} & \frac{\partial u_2}{\partial x_3} \\
			\frac{\partial u_3}{\partial x_1} & \frac{\partial u_3}{\partial x_2} & \frac{\partial u_3}{\partial x_3} 
		\end{array} \right] & = &
		 {\color[rgb]{0.5, 0.5, 0.5} \left[ \begin{array}{ccc}
				\frac{\partial u_1}{\partial x_1} 																  & 
				\frac{1}{2} \left( \frac{\partial u_1}{\partial x_2} + \frac{\partial u_2}{\partial x_1} \right)  &  
				\frac{1}{2} \left( \frac{\partial u_1}{\partial x_3} + \frac{\partial u_3}{\partial x_1} \right)  \\
				\frac{1}{2} \left( \frac{\partial u_1}{\partial x_2} + \frac{\partial u_2}{\partial x_1} \right)  &
				\frac{\partial u_2}{\partial x_2} 																  & 
				\frac{1}{2} \left( \frac{\partial u_2}{\partial x_3} + \frac{\partial u_3}{\partial x_2} \right)  \\
				\frac{1}{2} \left( \frac{\partial u_1}{\partial x_3} + \frac{\partial u_3}{\partial x_1} \right)  &
				\frac{1}{2} \left( \frac{\partial u_2}{\partial x_3} + \frac{\partial u_3}{\partial x_2} \right)  &
				\frac{\partial u_3}{\partial x_3} 
		\end{array} \right]} + \nonumber \\
		& &
		{\color[rgb]{0,0,1} \left[ \begin{array}{ccc}
				0																								  & 
				\frac{1}{2} \left( \frac{\partial u_1}{\partial x_2} - \frac{\partial u_2}{\partial x_1} \right)  &  
				\frac{1}{2} \left( \frac{\partial u_1}{\partial x_3} - \frac{\partial u_3}{\partial x_1} \right)  \\
				\frac{1}{2} \left( \frac{\partial u_2}{\partial x_1} - \frac{\partial u_1}{\partial x_2} \right)  &			0																								  &
				\frac{1}{2} \left( \frac{\partial u_2}{\partial x_3} - \frac{\partial u_3}{\partial x_2} \right)  \\
				\frac{1}{2} \left( \frac{\partial u_3}{\partial x_1} - \frac{\partial u_1}{\partial x_3} \right)  &
				\frac{1}{2} \left( \frac{\partial u_3}{\partial x_2} - \frac{\partial u_2}{\partial x_3} \right)  &
				0
		\end{array} \right]}	\nonumber
		}	
		\end{eqnarray}
		}
		rotation is anti-symmetric (see rotation matrix), strain part is symmetric.
		\uncover<4->{
		We generally define the deformation gradient tensor as:
		\begin{equation}
		\varepsilon_{ij} = \epsilon_{ij} + \omega_{ij} = 
					{\color[rgb]{0.5, 0.5, 0.5} \frac{1}{2} \left( \frac{\partial u_i}{\partial x_j} + \frac{\partial u_j}{\partial x_i} \right) dx_j} +
					{\color[rgb]{0,0,1} \frac{1}{2} \left( \frac{\partial u_i}{\partial x_j} - \frac{\partial u_j}{\partial x_i} \right) dx_j}\nonumber
		\end{equation}
		}
 
\end{frame}



\begin{frame}
	\frametitle{Strain and Rotation Tensors}
		{\bf Strain tensor} can be written: 
		\begin{eqnarray}
		\epsilon_{ij} & = & \frac{1}{2} \ \left( \frac{\partial u_i}{\partial x_j} + \frac{\partial u_j}{\partial x_i} \right) = 
								\left[ \begin{array}{ccc}
									\epsilon_{11} & \epsilon_{12} & \epsilon_{13} \\
									\epsilon_{21} & \epsilon_{22} & \epsilon_{23} \\
									\epsilon_{31} & \epsilon_{32} & \epsilon_{33}
								\end{array} \right]\nonumber
		\end{eqnarray} 
		symmetric, with 6 independent components since $\epsilon_{21} = \epsilon_{12}, \epsilon_{31} = \epsilon_{13}, \epsilon_{32} = \epsilon_{23}$\\[10pt]
		
		\uncover<2->{
		{\bf Rotation tensor} can be written:
		\begin{eqnarray}
		\omega_{ij} & = & \frac{1}{2} \ \left( \frac{\partial u_i}{\partial x_j} - \frac{\partial u_j}{\partial x_i} \right) = 
								\left[ \begin{array}{ccc}
									0             &   \omega_{12} & \omega_{13} \\
									- \omega_{12} & 0 			  & \omega_{23} \\
									- \omega_{13} & - \omega_{23} & 0
								\end{array} \right] \nonumber
		\end{eqnarray} 
		antisymmetric, with 3 independent components 
		}
\end{frame}

\begin{frame}
	\frametitle{Strain and Rotation from GPS Data}
	\begin{itemize}
		\item Can estimate all components of strain and rotation tensors directly from GPS data
		\item Equations in terms of 6 independent strain tensor components and 3 independent rotation tensor components
		\item \dots or in terms of the 9 components of the displacement gradient tensor $\varepsilon_{ij}$
		\item<2-> Write motions relative to reference site or reference point in terms of distance from reference (``remove translation''):
		\begin{eqnarray}
		u_i({\bf x_0} + {\bf dx_0}) - u_i({\bf x_0}) & = & \epsilon_{ij}dx_j + \omega_{ij} dx_j \nonumber		
		\end{eqnarray} 
		\item<2-> ${\bf x_0}$ is reference location, {\bf dx} is vector from reference to data location
	\end{itemize}		
\end{frame}

%\begin{frame}
%	\frametitle{2D Equations}
%	\begin{itemize}
%		\item Again, assume infinitesimal displacements 
%		\item Velocity $v$ as function of position ${\bf p}$ can be expanded into:
%		\begin{eqnarray}
%		v({\bf x} + {\bf dx}) & = & v ({\bf x}) + \frac{\partial v}{\partial {\bf x}} {\bf dx} \nonumber		
%		\end{eqnarray} 
%		\item with 
%		\begin{eqnarray}
%		v_x({\bf x} + {\bf dx}) & = & v_x ({\bf x}) + \frac{\partial v_x}{\partial x} dx + \frac{\partial v_x}{\partial y} dy\nonumber		\\
%		v_y({\bf x} + {\bf dx}) & = & v_y ({\bf x}) + \frac{\partial v_y}{\partial x} dx + \frac{\partial v_y}{\partial y} dy\nonumber		
%		\end{eqnarray} 
%		\item<2-> So:
%		\begin{eqnarray}
%		v({\bf x} + {\bf dx}) & = & v ({\bf x}) + \nabla V \cdot {\bf dx} \nonumber		
%		\end{eqnarray} 
%		\item<2-> Where $\nabla V$ is velocity gradient tensor:
%		\begin{eqnarray}
%			\nabla V & = & 	\left[ \begin{array}{cc}
%								\frac{\partial v_x}{\partial x}  & \frac{\partial v_x}{\partial y}   \\
%								\frac{\partial v_y}{\partial x}  & \frac{\partial v_y}{\partial y} 
%								\end{array} \right] \nonumber
%		\end{eqnarray} 
%	\end{itemize}		
%\end{frame}

%\begin{frame}
%	\frametitle{2D Equations}
%	\begin{itemize}
%		\item Assuming 2 (GPS) sites separated by (small) distance {\bf dx}:
%		\begin{eqnarray}
%		v({\bf x} + {\bf dx}) 				& = & v ({\bf x}) +  \nabla V \cdot {\bf dx} \nonumber \\ 
%		v({\bf x} + {\bf dx}) - v ({\bf x}) & = &  \nabla V \cdot {\bf dx} \nonumber \\ 
%		v_i - v_j                           & = &  \nabla V \cdot {\bf dx} \nonumber 
%		\end{eqnarray} 
%		\item<2-> Expand to 3 sites, where 1 site is reference:
%		{\small 
%		\begin{eqnarray}
%		{\bf d}        & = & G{\bf m}\nonumber \\
%		velocity\_diff & = & position\_differences \cdot velocity\_gradients \nonumber\\
%		\left[ \begin{array}{c}
%		v_{x_2} - v_{x_1}  \\
%		v_{y_2} - v_{y_1}  \\
%		v_{x_3} - v_{x_1}  \\
%		v_{y_3} - v_{y_1}  \\
%		\end{array} \right]   & = & 
%										\left[ \begin{array}{cccc}
%										x_2-x_1  & y_2-y_1 &  0       & 0       \\
%										0        & 0       &  x_2-x_1 & y_2-y_1 \\
%										x_3-x_1  & y_3-y_1 &  0       & 0       \\
%										0        & 0       &  x_3-x_1 & y_3-y_1 \\
%										\end{array} \right] 
%										\cdot 
%										\left[ \begin{array}{c}
%											\partial v_x / \partial x \\
%											\partial v_x / \partial y \\
%											\partial v_y / \partial x \\
%											\partial v_y / \partial y \\
%										\end{array} \right] \nonumber 
%		\end{eqnarray} 
%		}
%		\item<2-> Solve for velocity gradients via least squares
%	\end{itemize}		
%\end{frame}

%\begin{frame}
%	\frametitle{Strain Rates and Rotation Rates}
%	\begin{itemize}
%		\item displacement contains strain and rigid body rotation
%		\item \dots so velocity (displacement rate) contains strain rate and rotation rate
%		\item<2-> According to tensor theory, a second rank tensor can be decomposed into a symmetric and antisymmetric tensor:
%	\end{itemize}		
%	\uncover<2->{
%		{\footnotesize
%		\begin{eqnarray}
%		\nabla V &=& \frac{1}{2} \left[ \nabla V + \nabla V^{T} \right] + \frac{1}{2} \left[ \nabla V - \nabla V^{T} \right] \nonumber \\
%		         &=& ??? \nonumber \\
%		\uncover<3->{
%				 &=& \left[ \begin{array}{cc}
%			\frac{\partial v_x}{\partial x}  				& \frac{1}{2} \left( \frac{\partial v_x}{\partial y} + \frac{\partial v_y}{\partial x} \right ) \\
%			\frac{1}{2} \left( \frac{\partial v_x}{\partial y} + \frac{\partial v_y}{\partial x} \right )		&  \frac{\partial v_y}{\partial y} 
%						\end{array} \right] + 
%						\left[ \begin{array}{cc}
%			0  				& \frac{1}{2} \left( \frac{\partial v_x}{\partial y} - \frac{\partial v_y}{\partial x} \right ) \\
%			- \frac{1}{2} \left( \frac{\partial v_x}{\partial y} - \frac{\partial v_y}{\partial x} \right )		&  0
%						\end{array} \right] \nonumber \\
%		}
%		\uncover<4->{
%				 &=& \left[ \begin{array}{cc}
%				 		\dot\varepsilon_{11} & \dot\varepsilon_{12} \\
%				 		\dot\varepsilon_{21} & \dot\varepsilon_{22}
%						\end{array} \right] + 
%						\left[ \begin{array}{cc}
%				 		0 					& \dot\omega \\
%				 		- \dot\omega 	& 0
%						\end{array} \right] \nonumber		
%		}
%		\end{eqnarray} 
%		}
%	}
%\end{frame}

%\begin{frame}
%	\frametitle{Now: Strain Rates and Rotation Rates}
%	\begin{itemize}
%		\item we're doing this, because we deal with velocities a lot
%			\begin{eqnarray}
%			\nabla V &=& \left[ \begin{array}{cc}
%					 		\dot\varepsilon_{11} & \dot\varepsilon_{12} \\
%					 		\dot\varepsilon_{21} & \dot\varepsilon_{22}
%							\end{array} \right] + 
%							\left[ \begin{array}{cc}
%					 		0 					& \dot\omega \\
%					 		- \dot\omega 	& 0
%							\end{array} \right] \nonumber		
%			\end{eqnarray} 
%		\item Recall: 
%			\begin{eqnarray} 
%			v({\bf x} + {\bf dx}) & = & v ({\bf x}) + \nabla V \cdot {\bf dx} \nonumber		
%			\end{eqnarray} 
%		\item 2D case: 
%			\begin{eqnarray} 
%			v_x & = & V_x + \dot\varepsilon_{xx}\Delta x + \dot\varepsilon_{xy} \Delta y + \dot\omega \Delta y \nonumber \\
%			v_y & = & V_y + \dot\varepsilon_{xy}\Delta x + \dot\varepsilon_{yy} \Delta y - \dot\omega \Delta x \nonumber
%			\end{eqnarray} 
%		\item 2D case: 4 parameters to solve for (3 strain, 1 rotation), need $\geq$ 2 sites with horizontal data
%		\item 3D case: 9 parameters to solve for (6 strain, 3 rotation), need $\geq$ 3 sites with 3D data
%		\item keep in mind that this applies to displacements, too (``drop the rate'')
%	\end{itemize}
%\end{frame}

\begin{frame}
	\frametitle{Example: Strain from 3 GPS sites}
	\begin{columns}
		\column{0.5\textwidth}
			\begin{itemize}
				\item simple, general way to calculate average strain and rotation from 3 GPS sites
				\item (average strain for the area enclosed by the 3 sites)
				\item with more than 3 sites: divide network into triangles 
				\item Delaunay triangulation implemented in GMT is a quick way to do so
			\end{itemize}
		\column{0.5\textwidth}
			\begin{center}
				\includegraphics[width=\textwidth]{/home/roon/work/teaching/2015/Geodetic_methods/figures/lecture21_strain_example.png}
			\end{center}
			\begin{flushright}
				\tiny{\emph{J. Freymueller}}
			\end{flushright}
	\end{columns}
\end{frame}

\begin{frame}
	\frametitle{Example: Strain from 3 GPS sites}
	\begin{center}
		\includegraphics[width=0.3\textwidth]{/home/roon/work/teaching/2015/Geodetic_methods/figures/lecture21_strain_example.png}
	\end{center}
	\begin{flushright}
		\tiny{\emph{J. Freymueller}}
	\end{flushright}
	\vspace{-1cm}
	Let's look at this for a single baseline (horizontal deformation):
	\begin{eqnarray}
		\left[ \begin{array}{c}
			\Delta u_{1}^{AB} \\
			\Delta u_{2}^{AB}
		\end{array} \right] & = & 
						\left[ \begin{array}{c}
					 		\epsilon_{11}\Delta x_{1}^{AB} + \epsilon_{12}\Delta x_{2}^{AB} + \omega_{12} \Delta x_{2}^{AB}	\\
					 		\epsilon_{12}\Delta x_{1}^{AB} + \epsilon_{22}\Delta x_{2}^{AB} - \omega_{12} \Delta x_{1}^{AB}
						\end{array} \right] \nonumber \\
							& = & 			
			\uncover<2->{
							\left[ \begin{array}{cccc}
			\uncover<4->{			
					 		\Delta x_{1}^{AB} & \Delta x_{2}^{AB} & 0                 &    \Delta x_{2}^{AB}	\\
					 		0                 & \Delta x_{1}^{AB} & \Delta x_{2}^{AB} &  - \Delta x_{1}^{AB}    
			}
						\end{array} \right] \cdot 
						\left[ \begin{array}{c}
			\uncover<3->{			
					 		\epsilon_{11} \\
							\epsilon_{12} \\ 
							\epsilon_{22} \\ 
							\omega_{12}       
			}
						\end{array} \right] \nonumber \\
			}			
						& = & G \cdot {\bf m} \nonumber	
			\end{eqnarray} 	

\end{frame}

\begin{frame}
	\frametitle{Example: Strain from 3 GPS sites}
	\begin{center}
		\includegraphics[width=0.3\textwidth]{/home/roon/work/teaching/2015/Geodetic_methods/figures/lecture21_strain_example.png}
	\end{center}
	\begin{flushright}
		\tiny{\emph{J. Freymueller}}
	\end{flushright}
	\vspace{-1cm}
	Using all sites we get 4 equations in 4 unknowns:
	\begin{eqnarray}
		\left[ \begin{array}{c}
			\Delta u_{1}^{AB} \\
			\Delta u_{2}^{AB} \\
		\uncover<2->{
 			\Delta u_{1}^{AC} \\
			\Delta u_{2}^{AC}
		}
		\end{array} \right] & = & \left[ \begin{array}{cccc}
					 		\Delta x_{1}^{AB} & \Delta x_{2}^{AB} & 0                 &    \Delta x_{2}^{AB}	\\
					 		0                 & \Delta x_{1}^{AB} & \Delta x_{2}^{AB} &  - \Delta x_{1}^{AB}    \\
		\uncover<2->{
					 		\Delta x_{1}^{AC} & \Delta x_{2}^{AC} & 0                 &    \Delta x_{2}^{AC}	\\
					 		0                 & \Delta x_{1}^{AC} & \Delta x_{2}^{BC} &  - \Delta x_{1}^{AC}    
		}
						\end{array} \right] \cdot 
						\left[ \begin{array}{c}
					 		\epsilon_{11} \\
							\epsilon_{12} \\ 
							\epsilon_{22} \\ 
							\omega_{12}       
						\end{array} \right] \nonumber \\
		\uncover<3->{
		\left[ \begin{array}{c}
	 		\epsilon_{11} \\
			\epsilon_{12} \\ 
			\epsilon_{22} \\ 
			\omega_{12}       
		\end{array} \right] & = & \left[ \begin{array}{cccc}
					 		\Delta x_{1}^{AB} & \Delta x_{2}^{AB} & 0                 &    \Delta x_{2}^{AB}	\\
					 		0                 & \Delta x_{1}^{AB} & \Delta x_{2}^{AB} &  - \Delta x_{1}^{AB}    \\
					 		\Delta x_{1}^{AC} & \Delta x_{2}^{AC} & 0                 &    \Delta x_{2}^{AC}	\\
					 		0                 & \Delta x_{1}^{AC} & \Delta x_{2}^{AC} &  - \Delta x_{1}^{AC}    
						\end{array} \right]^{-1} \cdot 
						\left[ \begin{array}{c}
							\Delta u_{1}^{AB} \\
							\Delta u_{2}^{AB} \\
				 			\Delta u_{1}^{AC} \\
							\Delta u_{2}^{AC}
						\end{array} \right] \nonumber
			}
			\end{eqnarray} 	

\end{frame}
\begin{frame}
	\frametitle{Example: SE Alaska}
	\begin{center}
		\includegraphics[width=\textwidth]{/home/roon/work/teaching/2015/Geodetic_methods/figures/lecture22_SE_AK_velocities_elliott.png}
	\end{center}
	\begin{flushright}
		\tiny{\emph{Julie Elliott}}
	\end{flushright}
	\vspace{-0.25cm}
\end{frame}

\begin{frame}
	\frametitle{Example: SE Alaska, Icy Bay}
	\begin{center}
		\includegraphics[width=0.75\textwidth]{/home/roon/work/teaching/2015/Geodetic_methods/figures/lecture22_yakutat_icy_bay_velocities_elliott.png}
	\end{center}
	\begin{flushright}
		\tiny{\emph{Julie Elliott}}
	\end{flushright}
	\vspace{-0.25cm}
	\begin{itemize}
		\item velocities relative to stable North America \emph{(Sella et al, 2007)}
		\item velocities corrected for GIA using model of \emph{Larson et al (2005)}
	\end{itemize}
\end{frame}

\begin{frame}
	\frametitle{Example: SE Alaska, Icy Bay}
	\begin{center}
		\includegraphics[width=\textwidth]{/home/roon/work/teaching/2015/Geodetic_methods/figures/lecture22_yakutat_icy_bay_strains_elliott.png}
	\end{center}
	\begin{flushright}
		\tiny{\emph{Julie Elliott}}
	\end{flushright}

\end{frame}

\end{document}
